
\documentclass[tikz,border=10pt]{standalone}
\usepackage{pgfplots}
\pgfplotsset{compat=newest}
\usetikzlibrary{positioning}
\usetikzlibrary{positioning,calc}

\begin{document}
\begin{tikzpicture}[
    node distance=1mm, 
    box/.style={
        minimum width=3cm, 
        minimum height=2cm, 
        align=center,
        draw
    },
    largebox/.style={
        minimum width=6cm, 
        minimum height=3cm, 
        align=center,
        draw
    },
    plus/.style={
        draw=none,
        font=\Large
    },
    equals/.style={
        draw=none,
        font=\Large
    }
]

    % Draw the first plot box and the actual plot with pgfplots
    \node[box] (out1) {};
    \begin{axis}[
        at={(out1.south west)}, anchor=south west, 
        width=3cm, height=2cm,
        scale only axis,
        axis lines=none, % Hide all axis lines
        ticks=none, % Hide ticks
    ]
        \addplot[no marks] file {data1.txt};
    \end{axis}

    % Place the plus sign
    \node[plus, right=0mm of out1] (plus1) {+};

    % Draw the second plot box
    \node[box, right=0mm of plus1] (out2) {};
    \begin{axis}[
        at={(out2.south west)}, anchor=south west, 
        width=3cm, height=2cm,
        scale only axis,
        axis lines=none, % Hide all axis lines
        ticks=none, % Hide ticks
    ]
        \addplot[no marks] file {data2.txt};
    \end{axis}

    % Place the second plus sign
    \node[plus, right=0mm of out2] (plus2) {+};

    % Draw the third plot box
    \node[box, right=0mm of plus2] (out3) {};
    \begin{axis}[
        at={(out3.south west)}, anchor=south west, 
        width=3cm, height=2cm,
        scale only axis,
        axis lines=none, % Hide all axis lines
        ticks=none, % Hide ticks
    ]
        \addplot[no marks] file {data3.txt};
    \end{axis}

    % Place the third plus sign
    \node[plus, right=0mm of out3] (plus3) {+};

    % Draw the fourth plot box
    \node[box, right=0mm of plus3] (out4) {};
    \begin{axis}[
        at={(out4.south west)}, anchor=south west, 
        width=3cm, height=2cm,
        scale only axis,
        axis lines=none, % Hide all axis lines
        ticks=none, % Hide ticks
    ]
        \addplot[no marks] file {data4.txt};
    \end{axis}

    % Place the fourth plus sign
    \node[plus, right=0mm of out4] (plus4) {+};

    % ... (previous boxes and plus signs)

    % Draw the last box for the bias term
    \node[box, right=0mm of plus4, minimum width=1.5cm] (bias) {bias};

    % Calculate the center position for the new plot
    % This requires the calc TikZ library
    \coordinate (center) at ($(bias)!0.5!(out1)-(0,3cm)$);

    % Place the equals sign to the left of the center
    \node[equals] (equals) at ([xshift=-4cm]center) {=};

    % Draw the out5 box centered
    \node[largebox] (out5) at ([xshift=-5mm]center) {};
    \begin{axis}[
        at={(out5.south west)}, anchor=south west, 
        width=6cm, height=3cm,
        scale only axis,
        axis lines=none, % Hide all axis lines
        ticks=none, % Hide ticks
    ]
        \addplot[no marks] file {data5.txt}; % Assuming there is a data5.txt
    \end{axis}

    % Move the label "out5" to the bottom right corner of the large box
    \node[anchor=north east, inner sep=2mm] at (out5.north east) {Final output};

\end{tikzpicture}
\end{document}
